\subsection{Importance of MQTT for Digital Twins}

The effectiveness of Digital Twins (DT) is heavily bound by the "communication thread" connecting the physical assets to its virtual counterpart\cite{ref_11}.
For a DT to function as dynamic virtual replica, it requires continuous, near real-time synchronization of state data\cite{ref_12}.

MQTT is particularly critical where multiple DTs, such as a fleet of warehouse robots, interact simultaneously.
Security and reliability in these massive deployments are often enhanced through machine learning approaches, which rely on the steady stream of data MQTT provides\cite{ref_13}.
The protocol's lightweight nature allows for high-frequency telemetry updates without saturating the network, which is esstential for maintaining the consistency of the twin\cite{ref_14}.
Recent visions for 6G-enabled DTs emphasize that delays in this synchronization layer can lead to desynchronization between the physical and virtual worlds, reducing the effectiveness of predictive maintenance and collision avoidance algorithms\cite{ref_15}.
Therefore, MQTT serves not just as a transport mechanism, but as foundational synchronization layer that enables modern industrial DTs.