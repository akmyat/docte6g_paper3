\subsection{Wireless Connectivity Options for Smart Warehouse DTs}

The connectivity layer of a Smart Warehouse Digital Twin relies on heterogeneous wireless networks to link mobile robots and sensors. The industry currently debates the trade-offs between Wi-Fi 6 (IEEE 802.11ax) and 5G New Radio (NR)\cite{ref_20}.

WiFi 6 offers cost-effective, high bandwidth connectivity suitable for bulk data transfer.
Validation studies in ns-3 show that under low network loads, Wi-Fi 6 achieves latencies comparable to 5G ($<$10 ms). 
However, its contention-based medium access control (CSMA/CA) introduces non-deterministic jitter in dense environments where multiple robots compete for channel access\cite{ref_21}.

5G NR: Designed with Ultra-Reliable Low-Latency Communication (URLLC), 5G NR is increasingly favored for mission-critical control. 
Although it may exhibit higher baseline latency due to frame structures and core network traversal, it offers deterministic scheduling suitable for time-sensitive networking (TSN)\cite{ref_22}.

Recent literature suggests a hybrid approach for Digital Twins: utilizing 5G NR for critical, safety-related control traffic (where bounded latency is required) and Wi-Fi 6 for high-bandwidth, non-critical data such as video surveillance feeds. 
Validating this hybrid architecture requires the rigorous simulation environment that only ns-3 can provide.