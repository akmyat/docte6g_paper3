\subsection{MQTT for Industrial IoT Systems}

In the realm of Industrial Internet of Things (IIoT), efficient data transmission is very important.
MQTT has emerged as the de facto standard application-layer protocol among legacy request-response protocols like HTTP and CoAP\cite{iot_survey}.
MQTT, which is standardized by OASIS, utilizes a publish-subscribe architecture that decouples data producers (sensors) from consumers (controllers), which enable highly scalable deployments in bandwidth-constrained environments\cite{mqttv311}\cite{mosquitto}.

A distinguishing feature of MQTT for Industrial control is its support for three distinct Quality of Service (QoS) levels, which allow system architects to balance latency against reliability:
\begin{itemize}
    \item \textbf{QoS 0 (At most once):} Minimal latency "fire-and-forget" delivery.
    \item \textbf{QoS 1 (At least once):} Guarantees delivery but may result in duplicate messages.
    \item \textbf{QoS 2 (Exactly once):} Ensures single delivery through a four-step handshake\cite{mqttv311}.
\end{itemize}

Recent performance evaluations indicate that while QoS 2 provides the reliability required for critical control but its protocol overhead can reduce throughput by up to 40\% compared to QoS\cite{ref_6}\cite{ref_7}.
Furthermore, when security layers like TLS/SSL are added to protect integrity, computational delays increase, challenging the strict timing requirements of real-time industrial applications\cite{ref_8}\cite{ref_9}.
For these reasons, optimizing MQTT configurations to minimize latency while maintaining security is a critical area of active research\cite{ref_10}.