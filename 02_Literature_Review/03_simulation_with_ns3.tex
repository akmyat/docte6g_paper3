\subsection{Network Simulation and Protocol Evaluation in ns-3}

Validating these complex interactions in physical testbeds is often cost-prohibitive and risky.
Network simulation offers a viable alternative for pre-deployment verification.
The ns-3 simulator is widely recognized in academic research due to its discrete-event engine and detailed models of the full TCP/IP stack \cite{ref_16}.

However, a significant gap exists in the current ns-3 ecosystem: the lack of a native, feature-complete MQTT model.
Previous contributions, such as those by Augusto, provided compiled binaries without source code, preventing meaningful modification, extension, or validation\cite{ref_17}.
Similarly, \cite{ref_18} reports an MQTT implementation in ns-3 targeted at smart building IoT applications, yet neither the source code nor binaries are publicly available.
A 2023 review of IoT simulators highlighted that while other tools like OMNeT++ offer MQTT modules, they lack the cross-layer simulation capabilities of ns-3, particularly regarding complex physical layer models for 5G and Wi-Fi\cite{ref_19}.
This necessitates the development of a transparent, open-source MQTT library capable of interacting deeply with ns-3's wireless stacks.