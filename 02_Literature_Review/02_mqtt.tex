%\subsection{MQTT}
\sloppy
Message Queuing Telemetry Transport (MQTT) is a lightweight, client-server publish subscribe messaging protocol, standardized by OASIS for internet of things (IoT) and machine-to-machine (M2M) communication \cite{mqttv311}.
It is designed for resource-constrained devices and low-bandwidth, high-latency networks, making it a foundational technology for IoT deployments in domains such as smart buildings, smart factories and smart warehouses.
Digital Twins (DT) Applications, virtual counterparts of physical assets, are typically built on top of these IoT infrastructures. 
MQTT plays a central role by enabling real-time data exchange between physical entities (e.g. sensors, robots, and actuators) and their virtual representations, thereby continuous monitoring, simulation, predictive analysis, and operational optimization.

For simulation and research, MQTT has been implemented in ns-3 network simulator, allowing researchers to model publish/subscribe architectures and their integration with realistic wireless technologies such as WiFi, LoRaWAN and 5G.
Simulating MQTT within ns-3 provides valuable insights for DT development: it enables quantification of key performance indicators (KPIs) such as synchronization delay, command success probability, reliability under congestion, and energy/network overhead prior to real-world deployment.

However, existing MQTT implementations for ns-3 remain limited.
In \cite{mqttns3}, the author has provided a compiled binary file (.so file) for the MQTT library in the ns-3 framework without source code, preventing meaningful modification, extension, or validation.
Similarly, \cite{salimee2023mqttns3} reports an MQTT implementation in ns-3 targeted at smart building IoT applications, yet neither the source code nor binaries are publicly available.
These limitations significantly hinder reproducibility and restrict the community's ability to perform practical scenario-driven studies.

To address this gap, this research develops a fully open-source MQTT library for ns3.
The implementation is evaluated through Digital Twin Application scenarios in smart-warehouse conditions, demonstrating its suitability for end-to-end DT studies and enabling reproducible experimentation for future researchers.