\subsection{Scenario Description}

The smart warehouse scenario illustrated in Figure \ref{fig:smart_warehouse_scenario} depicts a cyber-physical system where various interconnected components work together to manage  the warehouse operations efficiently.
The \textit{controller} serve as the central hub for coordinating activities, sending commands, storing data, and maintaining overall situational awareness.
\textit{Package sensors} are placed at the conveyor belts that receive incoming packages, detecting their arrival and send notifications to the controller when new package arrives.
When a new package arrives, the controller checks if there is any available storage space in \text{racks} by sending a request to it and receiving the status update.
It also verifies that there are sufficient counts of \textit{mobile robots} available to pickup and transport the packages to the designated storage locations.
Once the controller confirms the availability of both storage space and mobile robots, it dispatches commands to the mobile robots to pick up the packages from the conveyor belt and deliver them to the assigned racks.
Whenever the controller receives the request to retrieve a package, it checks availability of mobile robots and sends command to the robot to fetch the package from the specified rack and deliver it to the \textit{drop zone} and update the status of the racks accordingly.
The environment sensors such as temperature and humidity sensors continuously monitor the warehouse conditions and send periodic updates to the controller to ensure optimal storage environment.
The video cameras positioned throughout the warehouse stream real-time footage to the controller for safety monitoring and surveillance.

\begin{figure}[ht]
    \centering
    \includegraphics[width=0.8\linewidth]{images/smart_warehouse_layout.png}
    \caption{Smart Warehouse Digital Twin Scenario}
    \label{fig:smart_warehouse_scenario}
\end{figure}

The components in the smart warehouse Digital Twin system and their interactions are illustrated in the class diagram in Figure \ref{fig:class_diagram_smart_warehouse_scenario}.
\begin{figure}[ht]
    \centering
    \includegraphics[width=0.8\linewidth]{images/smart_warehouse_DT_class_diagram.png}
    \caption{Class Diagram of Smart Warehouse Digital Twin System}
    \label{fig:class_diagram_smart_warehouse_scenario}
\end{figure}

These components generate diverse traffic types:
\begin{itemize}
    \item Robot control commands (e.g. destination update, pick and place instructions), which require ultra-low latency ($<10$--$20$~ms) and high relaiblity, as they influence mechanical actuatio and safety.
    \item Status updates from environmental sensors, package sensors, and rack availability sensors, which produce periodic traffic with moderate latency requirements ($<100$--$200$~ms).
    \item Video streams from fixed cameras, which generate high-bandwidth data flow with tighter jitter constraints ($<150$--$300$~ms) to support real-time monitoring.
\end{itemize}
Together, these heterogeneous traffic profiles reflect realistic operational conditions in a smart warehouse DT system.