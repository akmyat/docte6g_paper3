\subsection{MQTT ns-3 Module}

To support native publish-subscribe communication within the ns-3 simulation environment, a complete implementation of the MQTT protocol version 3.1.1 was developed directly within ns-3 and is available at \url{https://gitlab.com/simulations9070736/mqtt.git}..
The module consists of two main components: the MQTT client and the MQTT broker, which together provide an end-to-end MQTT stack suitable for IoT and DT simulations.

The MQTT client implements the full set of MQTT control packets, including CONNECT, CONNACK, PUBLISH (QoS 0, 1, 2), SUBSCRIBE, SUBACK, UNSUBSCRIBE, UNSUBACK, and DISCONNECT.
The client maintiains session state, manages retransmissions and handshake procedures for QoS level 1 and 2.

The MQTT broker functions as the central coordination point, managing concurrent client sessions, decoding incoming control packets, performing topic-based routing, and executing the complete QoS handshake logic, including PUBACK, PUBREC/PUBREL, AND PUBCOMP sequences to guarantee exactly-once delivery.
In addition, the broker implements mechanisms for client authentication, verifying CONNECT requests before establishing sessions and rejecting malformed or unauthorized connections.
It also maintains session persistence, storing subscription information, pending QoS messages (both level 1 and 2) to ensure that message delivery is correctly resumed after client reconnection.

Together, these components provide a comprehensive and extensible MQTT implementation that can operate over any ns-3 transport or wireless technology.
This integrated design enables systematic evaluation of publish-subscribe workloads in IoT and DT scenarios, leveraging ns-3's rich networking models and simulation capabilities.