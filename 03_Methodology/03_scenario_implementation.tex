\subsection{Smart Warehouse Digital Twin Scenario Implementation using MQTT}

The Smart Warehouse Digital Twin scenario is implemented using the MQTT protocol to facilitate communication between various components of the warehouse system. 
The implementation consists of three main subsystems: Inventory Management System, Environmental Monitoring System, and Robot Management System. 
Each subsystem utilizes MQTT topics to publish and subscribe to relevant data.

\textbf{Environmental Monitoring System:} This subsystem illustrated in Figure \ref{fig:environmental_monitoring_system}, environmental parameters such as temperature and humidity within the warehouse.
All the environmental sensors register to controller using the topic \textit{/warehouse/sensor/[sensor\_name]/register} where the controller subscribe and listening on the topic \textit{/warehouse/sensor/[sensor\_name]/register}.
Temperature and humidity sensor periodically publish their reading to the topic \textit{/warehouse/sensor/[sensor\_name]/temperature} and \textit{/warehouse/sensor/[sensor\_name]/humidity} respectively. 
The controller subscribes to these topics to receive real-time updates on environmental conditions.
However, the camera sensor, only listen to the controller command on the topic \textit{/warehouse/sensor/[sensor\_name]/video/commands} to start or stop the video streaming.
The controller control the video stream of camera by publishing command message to the topic \textit{/warehouse/sensor/[sensor\_name]/video/commands/startStream} and \textit{/warehouse/sensor/[sensor\_name]/video/commands/stopStream}.
\begin{figure}[ht]
    \centering
    \includegraphics[width=\linewidth]{images/environmetal_monitoring_system.png}
    \caption{Environmental Monitoring System MQTT Implementation}
    \label{fig:environmental_monitoring_system}
\end{figure}

\textbf{Inventory Management System:} This subsystem illustrated in Figure \ref{fig:inventory_management_system}, is responsible for tracking and managing the inventory within the warehouse.
Same as in the environmental monitoring system, all the inventory sensors register to controller using the topic \textit{/warehouse/sensor/[sensor\_name]/register} where the controller subscribe and listening on the topic \textit{/warehouse/sensor/[sensor\_name]/register}.
The package sensor publish the notification message to the topic \textit{/warehouse/sensor/[sensor\_name]/package} whenever a package is arrived to warehouse.
The controller subscribes to this topic and assign to the available robot to pick up the package and place on the rack.
Both the rack sensor and dropzone sensor waits for the controller command on the topic \textit{/warehouse/rack/[sensor\_name]/commands} and \textit{/warehouse/dropzone/[sensor\_name]/commands} respectively.
When the robot reaches to the rack, the controller publish the command to the topic \textit{/warehouse/rack/[sensor\_name]/commands/addPackage} to update the capacity of the rack.
As soon as the dropzone receive request to retrieve the package, it notify the controller on the topic \textit{/warehouse/dropzone/[sensor\_name]/commands/retrievePackage} and controller assign robot to retrieve the package from the rack.
Similar to place package operation, when the robot reaches to the rack, the controller publish the command to the topic \textit{/warehouse/rack/[sensor\_name]/commands/removePackage} to update the capacity of the rack.
As soon as the robot  drop the package to the dropzone, the controller notify the customer or truck using the topic \textit{/warehouse/dropzone/[sensor\_name]/commands/dropPackage}.
\begin{figure}[ht]
    \centering
    \includegraphics[width=\linewidth]{images/inventory_management_system.png}
    \caption{Inventory Management System MQTT Implementation}
    \label{fig:inventory_management_system}
\end{figure}

\textbf{Robot Management System:} This subsystem illustrated in Figure \ref{fig:robot_management_system}, manages the fleet of robots operating within the warehouse.
All the robots register to controller using the topic \textit{/warehouse/robot/[robot\_name]/register} where the controller subscribe and listening on the topic \textit{/warehouse/robot/[robot\_name]/register}.
All the PICKUP, PLACE, RETRIEVE, and DROP operation are assigned to the robot as task by the controller through the topic \textit{/warehouse/robot/[robot\_name]/command/assignTask}.
When the robot complete the assigned task, it notify the controller by publishing message to the topic \textit{/warehouse/robot/[robot\_name]/status/taskCompleted}.

\begin{figure}[ht]
    \centering
    \includegraphics[width=\linewidth]{images/robot_management_system.png}
    \caption{Robot Management System MQTT Implementation}
    \label{fig:robot_management_system}
\end{figure}
