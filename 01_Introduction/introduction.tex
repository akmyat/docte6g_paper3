\section{Introduction}

Industry 4.0 has transformed traditional logistics into Smart Warehouses. 
In these environments, Cyber-Physical Systems (CPS) and Digital Twins (DT) manage complex tasks with very little human help. 
A Digital Twin is a virtual replica of a physical system. 
To work correctly, it must stay in sync with physical assets—such as mobile robots, and sensors—in near real-time. 
The performance of a Digital Twin depends entirely on the connection between the virtual model and the real world. 
If there is a delay or data is lost, the virtual model will not match reality. 
This "desynchronization'' can cause failures in safety controls and maintenance systems.

To keep these systems synchronized, the Message Queuing Telemetry Transport (MQTT) protocol has become the standard for the Industrial Internet of Things (IIoT). 
MQTT is lightweight and uses a "publish-subscribe'' model. 
This allows sensors to send data efficiently without needing a constant, direct link to the central controllers. 
However, depending on underlying network infrastructure, MQTT performance can vary widely.
Modern warehouses can be implemented with different types of network such as wired Ethernet, Wi-Fi, and 5G New Radio (NR).
These networks have different characteristics in terms of latency, reliability, and coverage.
Therefore, it is essential to test these communication networks before building them to verify that it meet the requirement of Digital Twin applications, save money and reduce safety risks.

Network simulation is the best way to verify these systems before deployment. 
The ns-3 simulator is a widely respected tool in academic research because it models networks very accurately. 
However, ns-3 currently has a major gap: it lacks a native, open-source model for MQTT. 
Previous attempts to add MQTT to ns-3 often used closed code that researchers could not change or extend. 
This makes it hard to properly test how application protocols perform over modern wireless networks like 5G and Wi-Fi 6.

To address this research gap by providing a complete, open-source MQTT v3.1.1 module for ns-3. 
Using this new module, we created a realistic Smart Warehouse scenario to test the protocol's performance. 
Our main contributions are:
\begin{enumerate}
    \item \textbf{Development of a Native ns-3 MQTT Module:} We built a transparent MQTT Client and Broker that supports all Quality of Service (QoS) levels (0, 1, and 2). 
    This allows for deep analysis within the simulator.
    
    \item \textbf{Smart Warehouse Scenario Design:} We designed a realistic industrial environment with different types of data traffic, including fast robot control commands, regular sensor updates, and heavy video streams.
    
    \item \textbf{Comparative Network Analysis:} We systematically tested MQTT performance over Ethernet, Wi-Fi 6, and 5G NR. 
    By measuring speed and reliability across these networks, we provide real data to help industries choose the best connection for their needs.
\end{enumerate}

The rest of this paper is organized as follows: Section II reviews previous work on IIoT protocols and simulation. 
Section III explains our methodology, including the new ns-3 MQTT module and the warehouse scenario. 
Section IV presents the results of our simulations. 
Finally, Section V concludes the study and suggests future research.
