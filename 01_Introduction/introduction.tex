\section{Introduction}

The rapid adoption of Industry 4.0 principles has transformed conventional warehouse operations into large-scale Cyber-Physical Systems (CPS) where Digital Twins (DTs) continuously interact with fleets of robots, sensors, and automated storage infrastructure. A DT provides a virtual representation of physical processes, and its usefulness depends critically on the timeliness and reliability of the communication link that synchronizes state between both domains. Even small disruptions or delays may result in desynchronization, leading to degraded operational efficiency or safety risks in autonomous systems.

MQTT has emerged as the de facto industrial messaging protocol supporting this synchronization layer due to its lightweight publish–subscribe model and configurable Quality of Service (QoS) semantics. However, MQTT behavior is tightly coupled to the characteristics of the underlying network—wired Ethernet, Wi-Fi, or emerging 5G NR—and choosing the appropriate connectivity technology is non-trivial for latency-sensitive and reliability-critical DT workloads. Evaluating these tradeoffs in real deployments is costly and impractical, making high-fidelity network simulation essential.

Despite the prominence of ns-3 as a platform for cross-layer network evaluation, the simulator lacks a transparent and extensible MQTT implementation. Existing attempts provide only binary artifacts or partial functionality, preventing researchers from studying MQTT interactions with modern wireless stacks. This absence restricts realistic DT evaluations, particularly when heterogeneous traffic types—robot-control commands, periodic telemetry, and video streams—need to be jointly assessed.

To address this limitation, we develop a full MQTT v3.1.1 stack for ns-3, including client and broker applications supporting QoS 0/1/2, session persistence, retransmission logic, and topic-based routing. Using this module, we design a Smart Warehouse DT scenario that captures representative operational dynamics and evaluate MQTT performance across Ethernet, Wi-Fi 6, and 5G NR. The contributions of this paper are:

\begin{enumerate}
    \item \textbf{Native MQTT Implementation for ns-3:} 
    A fully open-source MQTT client and broker supporting all control packets, QoS semantics, and session management, is implemented. 
    
    \item \textbf{Smart Warehouse DT Design:} A realistic model of heterogeneous traffic flows—robot control, environmental telemetry, and inventory operations—reflective of industrial automation requirements, is presnted.
    
    \item \textbf{Comparative Multi-Technology Evaluation:} A systematic performance study of MQTT over Ethernet, Wi-Fi 6, and 5G NR, is presented, highlighting how network characteristics influence DT synchronization and reliability.
\end{enumerate}

The remainder of the paper reviews relevant literature, describes the MQTT module and simulation methodology, presents the comparative results, and discusses implications for real-world DT deployments.