As for the next step, we want to analyze the performance of traffic types that are generated in Smart Warehouse Digital Twin Application.
All the communication between the digital twin components are carried through MQTT protocol over Ethernet.
The MQTT messages exchanged between the components can be categorized into three types: telemetry, operations, and control messages.
Telemetry messages include periodic updates from temperature and humidity sensors.
Operations messages consist of commands sent between controllers and package detection sensors, racks and dropzone.
Control messages are commands sent from the controller to moving robots to manage their navigation and tasks within the warehouse.
The average payload sizes for all types of messages can be seen in Table \ref{tab:average_payload_sizes}.

\begin{table}[ht!]
    \caption{Average Payload Sizes of MQTT Messages in Smart Warehouse Digital Twin Application}
    \begin{tabular}{|l|r|}
        \hline
        \textbf{MQTT Message Type} & \textbf{Average Payload Size (bytes)} \\
        \hline
        Temperature telemetry&  81 \\
        \hline
        Humidity telemetry &  81 \\
        \hline
        Package detection operation &  158 \\
        \hline
        Rack Add Package operation &  119 \\
        \hline
        Rack Remove Package operation &  119 \\
        \hline
        Dropzone Retrieve Package operation &  100 \\
        \hline
        Dropzone Drop Package operation &  94 \\
        \hline
        Robot Assign Task control &  112 \\
        \hline
        Robot Task Complete control &  37 \\
        \hline
    \end{tabular}
    \label{tab:average_payload_sizes}
\end{table}

The performance of these three types of traffic over Ethernet, WiFi, and 5G networks are analyzed in the following subsections in order to figure out the delay and packet loss experienced by each type of message in different network setups.

\subsubsection{Deployment over Ethernet Connectivity}

The performance of the three types of MQTT messages over Ethernet connectivity is analyzed in this subsection.
Three different Ethernet speeds (100 Mbps, 1 Gbps, and 10 Gbps) are considered for the analysis.
Different level of MQTT QoS levels (0, 1, and 2) are also taken into account while analyzing the performance.

The results for telemetry messages over Ethernet connectivity are shown in Figure~\ref{fig:ethernet_telemetry_results}.
As the level of QoS increases, the average delay increases from $47 \mu s$, $75 \mu s$, to $207 \mu s$ for 100 Mbps Ethernet speed.
This is due to the additional overhead for ensuring message delivery.
It can be observed that as the Ethernet speed increases, the average delay experienced by telemetry messages decreases from $47 \mu s$, $45 \mu s$, to $37 \mu s$ for QoS level 0.
But for QoS levels 1 and 2, the reduction of average delay is significantly small when the Ethernet speed increases from 1 Gbps to 10 Gbps.

\begin{figure}[ht!]
    \centering
    \includegraphics[width=\linewidth]{images/ethernet_telemetry_results.png}
    \caption{Ethernet Telemetry Results for Smart Warehouse Digital Twin Application}
    \label{fig:ethernet_telemetry_results}
\end{figure}

The results for operations messages over Ethernet connectivity are shown in Figure~\ref{fig:ethernet_operations_results}.
The trend of average delay decreases with increasing Ethernet speed and increases with higher QoS levels is similar to the telemetry messages.
Among the messages Rack Remove Package messages experiences packet loss for QoS level 0 due to the non-reliable delivery nature of this QoS level but using higher QoS levels (1 and 2) ensures reliable delivery of these messages without any packet loss.
Under QoS 0, most operations fall below $30-90 \mu s$ delay for all three Ethernet speeds.
Under QoS 1 and 2, the delays increase significantly due to the acknowledgment and retransmission mechanisms yet benefit from higher speeds, notably for package detection messages.
QoS 2 exhibits the highest delays, reach $>500 \mu s$ at 100 Mbps, decreasing below $200 \mu s$ by 10 Gbps, showing that although reliability mechanisms add overhead, sufficient link capacity can mitigate queuing effects for operational traffics.

\begin{figure}[ht!]
    \centering
    \includegraphics[width=\linewidth]{images/ethernet_operations_results.png}
    \caption{Ethernet Operations Results for Smart Warehouse Digital Twin Application}
    \label{fig:ethernet_operations_results}
\end{figure}

The results for control messages over Ethernet connectivity are shown in Figure~\ref{fig:ethernet_control_results}.
Among the control messages, the assign task message have higher payload size (112 bytes) compared to the complete task message (37 bytes).
Under QoS 0, the difference of average delay between these two messages is not significant across all three Ethernet speeds.
However, under QoS levels 1 and 2, the assign task message experiences higher average delay compared to the complete task message due to the larger payload size.
Only QoS 0 ensures that average delay remain below $20 ms$ for both control messages across all three Ethernet speeds while QoS levels 1 and 2 were not able to meet this requirement as average delays exceed $70 ms$ for assign task messages.
Higher delays are observed because the robots are moving around the warehouse and connectivity degradations may occur, leading to retransmissions and increased latency.
As QoS level 0 does not guarantee reliable delivery of messages, deployment over ethernet connectivity does not meet the requirements for control messages in smart warehouse digital twin application.

\begin{figure}[ht!]
    \centering
    \includegraphics[width=\linewidth]{images/ethernet_control_results.png}
    \caption{Ethernet Control Results for Smart Warehouse Digital Twin Application}
    \label{fig:ethernet_control_results}
\end{figure}

\subsubsection{Deployment over WiFi Connectivity}

The performance of the three types of MQTT messages over WiFi connectivity is analyzed in this subsection.
Three different WiFi standards (802.11n, 802.11ac, and 802.11ax) are considered for the analysis.
Different level of MQTT QoS levels (0, 1, and 2) are also taken into account while analyzing the performance.

The Figure \ref{fig:wifi_telemetry_results} depicts the results for telemetry messages over WiFi connectivity.
Compared to the deployment over Ethernet connectivity, the average delay experienced by telemetry messages over WiFi connectivity is significantly higher due to the shared medium access nature of WiFi.
Similart to the Ethernet connectivity, as the QoS level increases, the average delay also increases and decreases as the WiFi standard improves from 802.11n to 802.11ax.
Under QoS 0, delays decline from ~1.28--1.30 ms with 802.11n to ~1.11--1.12 ms under 802.11ax, reflecting efficiency gains for small payload messages.
With QoS 1, telemetry delays increase to approximately 1.57–1.78 ms, with only minor improvements across Wi-Fi generations, while QoS 2 further elevates latency to ~1.82–1.97 ms, effectively saturating performance gains from PHY upgrades.
Under QoS level 2, the average delay experience by temperature messages is slightly more than humidity messages even though both messages have the same payload size (81 bytes).

\begin{figure}[ht!]
    \centering
    \includegraphics[width=\linewidth]{images/wifi_telemetry_results.png}
    \caption{WiFi Telemetry Results for Smart Warehouse Digital Twin Application}
    \label{fig:wifi_telemetry_results}
\end{figure}


The Figure \ref{fig:wifi_operations_results} depicts the results for operations messages over WiFi connectivity.
For QoS 0, delays significantly drop from ~2.5 ms on 802.11n to ~2.05-2.07 ms on 802.11ac/ax for pakcage detection messages while maintaining sub-milisecond values $(<0.8 ms)$ for other operation messages.
At QoS 2, the delay overhead becomes more significant, with package detection delays rising to ~3.5 ms on 802.11n and stabilizing around ~3.1 ms on 802.11ax while other operational messages remain between 0.85-1.3 ms.
These results demonstrate that WiFi improvements modestly reduce queuing and transmission delays, particularly for heavy operational workloads, but cannot fully compensate for protocol-level acknowledgement overhead at higher MQTT QoS levels.

\begin{figure}[ht!]
    \centering
    \includegraphics[width=\linewidth]{images/wifi_operations_results.png}
    \caption{WiFi Operations Results for Smart Warehouse Digital Twin Application}
    \label{fig:wifi_operations_results}
\end{figure}

The Figure \ref{fig:wifi_control_results} depicts the results for control messages over WiFi connectivity.
Under QoS 0, delay decreases from approximately 1.0 ms on 802.11n to ~0.83 ms on 802.11 ac/ax, indicating that newer WiFi standards help moderately reduce delay through higher data rates and better channel efficiency.
For QoS 1 and QoS 2, however, improvement remain limited: delays stays around 1.1-1.3 ms for assign task messages and 0.7-1.0 ms for task completion messages, with only marginal reductions under 802.11ax.
Unlinke what was obeserved in ethernet connectivity, the delays are below 2 ms even at QoS 2 for assign task messages, likely due to lower baseline delays in WiFi.

\begin{figure}[ht!]
    \centering
    \includegraphics[width=\linewidth]{images/wifi_control_results.png}
    \caption{WiFi Control Results for Smart Warehouse Digital Twin Application}
    \label{fig:wifi_control_results}
\end{figure}

\subsubsection{Deployment over 5G NR Connectivity}

The performance of the three types of MQTT messages over 5G NR connectivity is analyzed in this subsection.
The 5G deployment utilizes a standalone (SA) architecture with a 3.5 GHz frequency band.
Different level of MQTT QoS levels (0, 1, and 2) are also taken into account while analyzing the performance.

The Figure \ref{fig:5g_telemetry_results} depicts the results for telemetry messages over 5G NR connectivity.
Across all QoS levels, measured delays remain nearly identical at approximately 23.46 m, indicating negligible sensitivity to QoS-induced overhead for small telemetry payloads in this 5G setup.
This unusually flat performance contrasts with Ethernet and Wi-Fi results, where acknowledgment overhead clearly increased latency.
In the 5G case, the dominant contributors to delay are radio scheduling latency, frame structure timing (TTI granularity), and core network traversal, which overshadow the incremental overhead introduced by MQTT acknowledgments for small payloads. As a result, improvements or changes in MQTT QoS levels do not significantly alter telemetry performance under 5G NR.

\begin{figure}[ht!]
    \centering
    \includegraphics[width=\linewidth]{images/5G_telemetry_results.png}
    \caption{5G Telemetry Results for Smart Warehouse Digital Twin Application}
    \label{fig:5g_telemetry_results}
\end{figure}

The Figure \ref{fig:5g_operations_results} depicts the results for operations messages over 5G NR connectivity.
Package detection messages exhibits the highest latency, ranging from approximately 59 ms (QoS 0) to 55 ms (QoS 2).
Other operation messages remain clustered around 22–34 ms, with only minor variations across QoS levels.
Notably, increasing QoS does not consistently increase latency; in some cases QoS 2 achieves slightly lower delays than QoS 0, likely due to improved packet delivery reliability, fewer retransmission timeouts at the RLC/MAC layer, and more stable traffic scheduling.
Overall, the results indicate that radio access and scheduling processes dominate latency performance for operation traffic, while MQTT acknowledgment overhead remains secondary.

\begin{figure}[ht!]
    \centering
    \includegraphics[width=\linewidth]{images/5G_operations_results.png}
    \caption{5G Operations Results for Smart Warehouse Digital Twin Application}
    \label{fig:5g_operations_results}
\end{figure}

The Figure \ref{fig:5g_control_results} depicts the results for control messages over 5G NR connectivity.
Task assignment delays ranges between ~31.7 ms (QoS 0) and ~39.2 ms (QoS 1/2), while task completion messages consistently remain lower at approximately 25–26 ms across all QoS levels.
The modest increase for assignment commands under higher QoS reflects the additional acknowledgment exchanges but remains comparatively small relative to the baseline radio access and scheduling delays inherent to the 5G system. Importantly, the differences between QoS levels are limited to single-digit milliseconds, confirming that control message latency is primarily constrained by wireless air-interface and core-network delays rather than MQTT protocol semantics.
For real-time robotic coordination, this suggests that reliability-enhancing QoS levels can be safely employed over 5G NR without severely degrading control responsiveness.

\begin{figure}[ht!]
    \centering
    \includegraphics[width=\linewidth]{images/5G_control_results.png}
    \caption{5G Control Results for Smart Warehouse Digital Twin Application}
    \label{fig:5g_control_results}
\end{figure}