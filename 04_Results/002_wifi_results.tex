\subsubsection{Deployment over WiFi Connectivity}

The performance of the three types of MQTT messages over WiFi connectivity is analyzed in this subsection.
Three different WiFi standards (802.11n, 802.11ac, and 802.11ax) are considered for the analysis.
Different level of MQTT QoS levels (0, 1, and 2) are also taken into account while analyzing the performance.

The Figure \ref{fig:wifi_telemetry_results} depicts the results for telemetry messages over WiFi connectivity.
Compared to the deployment over Ethernet connectivity, the average delay experienced by telemetry messages over WiFi connectivity is significantly higher due to the shared medium access nature of WiFi.
Similart to the Ethernet connectivity, as the QoS level increases, the average delay also increases and decreases as the WiFi standard improves from 802.11n to 802.11ax.
Under QoS 0, delays decline from ~1.28--1.30 ms with 802.11n to ~1.11--1.12 ms under 802.11ax, reflecting efficiency gains for small payload messages.
With QoS 1, telemetry delays increase to approximately 1.57–1.78 ms, with only minor improvements across Wi-Fi generations, while QoS 2 further elevates latency to ~1.82–1.97 ms, effectively saturating performance gains from PHY upgrades.
Under QoS level 2, the average delay experience by temperature messages is slightly more than humidity messages even though both messages have the same payload size (81 bytes).

\begin{figure}[ht!]
    \centering
    \includegraphics[width=\linewidth]{images/wifi_telemetry_results.png}
    \caption{WiFi Telemetry Results for Smart Warehouse Digital Twin Application}
    \label{fig:wifi_telemetry_results}
\end{figure}


The Figure \ref{fig:wifi_operations_results} depicts the results for operations messages over WiFi connectivity.
For QoS 0, delays significantly drop from ~2.5 ms on 802.11n to ~2.05-2.07 ms on 802.11ac/ax for pakcage detection messages while maintaining sub-milisecond values $(<0.8 ms)$ for other operation messages.
At QoS 2, the delay overhead becomes more significant, with package detection delays rising to ~3.5 ms on 802.11n and stabilizing around ~3.1 ms on 802.11ax while other operational messages remain between 0.85-1.3 ms.
These results demonstrate that WiFi improvements modestly reduce queuing and transmission delays, particularly for heavy operational workloads, but cannot fully compensate for protocol-level acknowledgement overhead at higher MQTT QoS levels.

\begin{figure}[ht!]
    \centering
    \includegraphics[width=\linewidth]{images/wifi_operations_results.png}
    \caption{WiFi Operations Results for Smart Warehouse Digital Twin Application}
    \label{fig:wifi_operations_results}
\end{figure}

The Figure \ref{fig:wifi_control_results} depicts the results for control messages over WiFi connectivity.
Under QoS 0, delay decreases from approximately 1.0 ms on 802.11n to ~0.83 ms on 802.11 ac/ax, indicating that newer WiFi standards help moderately reduce delay through higher data rates and better channel efficiency.
For QoS 1 and QoS 2, however, improvement remain limited: delays stays around 1.1-1.3 ms for assign task messages and 0.7-1.0 ms for task completion messages, with only marginal reductions under 802.11ax.
Unlinke what was obeserved in ethernet connectivity, the delays are below 2 ms even at QoS 2 for assign task messages, likely due to lower baseline delays in WiFi.

\begin{figure}[ht!]
    \centering
    \includegraphics[width=\linewidth]{images/wifi_control_results.png}
    \caption{WiFi Control Results for Smart Warehouse Digital Twin Application}
    \label{fig:wifi_control_results}
\end{figure}