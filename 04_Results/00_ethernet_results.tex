\subsection{Comparison of QoS Metrics upon Varying Ethernet Speeds and Payload Sizes}

In order to verify that our implementation of MQTT protocol in NS-3 is functioning correctly,  we analyze the QoS metrics under different Ethernet speeds and payload sizes. 
We conduct simulations with Ethernet speeds of 10 Mbps, 100 Mbps, and 1 Gbps, while varying the payload sizes from 100 bytes to 10,000 bytes.
All of the three different levels of QoS (0, 1, and 2) are tested in the simulations. 

In MQTT protocol, based on the QoS levels, different ackowledgement messages are exchanged between the subscriber and the broker to ensure delivery of the messages.
These different acknowledgement messages introduce additional overhead in the communication, which can impact the latency and throughput of the system.
For the QoS level 0, there is no acknowledgement message exchanged between the subscriber and broker, resulting in the lowest latency and highest throughput.
The QoS level 1 introduces a PUBACK message from the broker to the subscriber, which adds some overhead and increases the latency while slightly reducing the throughput while ensuring message delivery at least once.
The QoS level 2 involves a four-step handshake process (PUBLISH, PUBREC, PUBREL, PUBCOMP) to guarantee that the message is delivered exactly once. 
This process introduces the highest overhead, resulting in the highest latency and lowest throughput among the three QoS levels.
The results of the simulations are presented in Figure~\ref{fig:ethernet_results}. The simulation results show that as the Ethernet speed increases, the latency decreases and throughput increases for all QoS levels.
Similarly, as the payload size increases, the latency increases and throughput decreases for all QoS levels. This reflects the expected behavior of MQTT protocol under different QoS levels.
A

\begin{figure}[ht]
    \centering
    \includegraphics[width=0.8\linewidth]{images/ethernet_result.png}
    \caption{Comparison of QoS Metrics upon Varying Ethernet Speeds and Payload Sizes}
    \label{fig:ethernet_results}
\end{figure}

\hd{Font sizes are too small in Fig. \ref{fig:ethernet_results}}